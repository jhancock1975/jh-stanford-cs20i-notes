\documentclass[a5paper,11pt]{article}

%for coloring cell in a table
\usepackage[table]{xcolor}% http://ctan.org/pkg/xcolor

\usepackage{amsmath}
\usepackage{amssymb}

% for proofs  environment
\usepackage{amsthm}

% for 3d plots
\usepackage{pgfplots}
\usepackage{pgfplotstable}
\usepgfplotslibrary{patchplots}

\usepackage[backend=bibtex]{biblatex}
\bibliography{notes}

% for probability trees
\usepackage{tikz}
\usetikzlibrary{trees}

% for Venn diagrams
\usetikzlibrary{shapes,backgrounds}

% for plots
\usepackage{ pgfplots}
% inserted on suggestion in warning during compilation
\pgfplotsset{compat=1.9}

%for strikethrough text
\usepackage{soul}

%for R source code listing
\usepackage{listings}

%for block quotes
\usepackage{csquotes}

%for Theorems & Lemmas
\newtheorem{thm}{Theorem}
\newtheorem{lem}[thm]{Lemma}

% For not indenting the first line of paragraphs:
\setlength{\parindent}{0pt}

% define the title
\author{John Hancock}
\title{CS 20SI Notes}

\begin{document}

% generates the title
\maketitle

% insert the table of contents
\tableofcontents

\section{References and License}

In this document we are recording notes on reading material in
\cite{cs20siIntroSlides}.

Please see the references section for detailed citation information.


\section{Starting TensorBoard}

Starting TensorBoard is relatively easy, though the initial documentation
may seem rather daunting.

We found all that is necessary is that one create a TensorFlow session,
then pass the location of a logging directory to TensorFlow's session
file writer.  Hence, one must run the following
\begin{lstlisting}
file_writer = tf.summary.FileWriter('/tmp/tensorboard', sess.graph)
\end{lstlisting}
If one already has TensorBoard installed on one's system then simply 
starting TensorBoard (with the, "tensorboard," command) will start a server.
The server startup message will include a URL that one can open in a browser.
If one has already run a graph in a tensorflow session object, TensorBoard
will render that graph.
We found the documentation in \cite{tensorBoardIntro} to be useful in
determining how to use TensorBoard to render a simple graph.
\printbibliography{}

\end{document}
